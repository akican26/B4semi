\documentclass[11pt,a4j,uplatex]{jsarticle}
\usepackage{ascmac}
\usepackage{amsmath}
\usepackage[dvipdfmx]{graphicx}
\usepackage{upgreek}
\usepackage{cases}%連立方程式

\newsavebox{\circlebox}
\savebox{\circlebox}{\fontencoding{OMS}\selectfont\Large\char13}
\newlength{\circleboxwdht}
\newcommand{\centercircle}[1]{%
  \setlength{\circleboxwdht}{\wd\circlebox}%
  \addtolength{\circleboxwdht}{\dp\circlebox}%
  \raisebox{0.4\dp\circlebox}{%
    \parbox[][\circleboxwdht][c]{\wd\circlebox}{\centering#1}}%
  \llap{\usebox{\circlebox}}%
}	%丸数字(文字)環境。\centercircle{入れたい文字} で丸文字を表示する。


\title{Semiconductor Optics 和訳}
\author{3.2~3.3}


\begin{document}

\maketitle %タイトル

\thispagestyle{empty}%このページにはページ番号を入れない.
\clearpage
\addtocounter{page}{-1}

\newpage

\tableofcontents %目次

\thispagestyle{empty}%このページにはページ番号を入れない
\clearpage
\addtocounter{page}{-1}

%\listoffigures%図目次。確認用。後で消す

\newpage
\section{3.2 微視的側面}
先行した単元とは対照的に、今回は、微視的観点からの、光と物質の間の基本的な対応過程を与える。
(ほとんどの場合、ガスのような希薄な系で十分な)光と物質の間の対応のために、摂動や、弱いカップリングアプローチをここで用いる。固体においては、強いカップリングアプローチが必要不可だ(それはポラリトンの概念を導く。チャプター5にて紹介する。)。セクション3.2.1では、吸収、自然放出、誘導放出と名付けられている、光と物質の間の基本的な対応過程について述べる。セクション3.2.2では、摂動理論の枠組みの中の、線形の光学性質の扱いに進む。それらのトピックは多くの本にて扱われているので([81M1,90K1,96Y1]のチャプター1や、[55S1,71F1,73H1,76H1,92M1]のチャプター2等を参照。)、ここでは詳細を述べる必要はない。
\subsection{3.2.1 吸収、自然放出、誘導放出と仮想励起}
簡単のため、図のような、一定の数の2準位"原子"を仮定する。すべての原子は、基底状態か励起状態のどちらかである1つの電子を持っている。後々この2準位系を、半導体におけるバンドにまで拡大するが、基本的な対応過程は同じである。
図(a)では、入射した光子が、基底状態の原子に衝突している。一定の確率で光子は消滅して、電子は励起状態へ遷移するのに十分なエネルギーを受け取る。エネルギー保存則から、光子は
式
の関係を満たす必要がある。ここで、(右辺)は励起状態と基底状態のエネルギー差である。この過程を吸収と呼ぶ。
光子のエネルギーがほかのエネルギーに変換された時、電子が散乱過程を受けた時、それは自身のコヒーレンスを破壊するか、あるいは、電子的な偏光の、もっと正確なコヒーレンスを入射光の遷移につなげる。チャプター23を見ること。電子はやがて基底状態に戻り、例えば熱のような光子振動や、入射光とコヒーレントでない光となってエネルギーを失う。前者を非放射遷移、後者を蛍光共鳴と呼ぶ。
もし入射光が、励起状態の電子を伴う原子に衝突したとき、一定の確率で、励起状態から基底状態への電子の遷移を誘発する。この過程においてでは、2つ目の光子は入射光と同じ運動量、エネルギー、偏光状態、位相で作られる。この過程は誘導放出あるいは刺激放出と呼ばれる。この過程は光場を増幅する。したがって、これはすべてのレーザー(Light Amplification by Stimulated Emission of Radiation:刺激放出による光の増幅)の基本の機構となる。吸収と誘導放出は密接な関係を持つ現象である(図(b))。
励起状態の電子は一定の確率で、光子を放出するか、フォノンや衝突としてエネルギーを失うか、のどちらかを自らの手で行うことで、基底状態に戻る。これは自然放出あるいは自然放射性再結合と呼ばれ、後者は非放射性再結合としても知られる。自然放出はほかの考え方で理解することもできる。セクション2.5で、式(2.54)に関連して、光子は調和振動に似ていて、結果としてゼロ点エネルギーを持つことを見た。このゼロ点エネルギーはすべての光子のモードに存在する。これは、調和振動はゼロ点以下のエネルギーを持てないことから、吸収されない。しかし、これは図(b)に関連して論じたように、遷移を誘発することができる。そのため、自然放出は電磁場のゼロ点振動(これは、電磁場の真空ゆらぎまたはゼロ点ゆらぎとも呼ばれる)によって誘発されたとみなすことができる。
最後の過程は仮想励起である。個温現象を理解することはしばしば学生たちにとっていくつかの問題を起こす。そのため、このトピックはゆっくり進行し、図(d)とともに、様々な観点、文脈からの説明を試みる。仮想励起は、励起状態と波動関数がおなじで、固有エネルギーが少しだけ違う状態が作られたことを意味する。この過程は、空間座標と運動量座標で書かれた量子力学の不確定性原理を通して可能となる。
式
同じような関係が、エネルギーと時間の間にも存在する。
式
ここで、式(b)が必要になる。この式は、上記の条件を満たす最大時間$\Delta t$まで、$\Delta E$だけエネルギー保存則が破壊できることが可能である、と言っている。%なんのこっちゃ
言い換えると、$\Delta E$の確かな精度でエネルギーを定義したいとき、その状態は最低$\Delta t$の間存在しなくてはならない。式(3.37b)はまた、単純な古典的な波の理論(音波のような)で有効で、フーリエ変換としてよく知られている。$\Delta t$の時間だけ続く中心振動数調和振動は式(3.37c)で与えられるスペクトル幅$\Delta \omega$をもっている。

\end{document}
