\documentclass[11pt]{jsarticle}%書体
\usepackage{ascmac}
\usepackage{amsmath}%数式
\usepackage[dvipdfmx]{graphics}%画像
\usepackage{upgreek}%ギリシャ文字立体
%\usepackage{name}

\title{タイトル}%タイトル
\author{著者}%著者
\date{日付}%日付

\begin{document}%本文はじめ
\maketitle%タイトル
\tableofcontents%目次
\listoffigures%図目次

\section{非結合発振器のアンサンブル}%セクション

物質の光学的性質は、物質中の様々な種類の振動子を電磁放射場に結合することによって決定されます。言い換えれば、入射電磁場はこれらの振動子に駆動振動または強制振動を行わせるであろう。これらの駆動された振動の振幅は、入射場の角周波数ω、振動子の固有振動数$\omega_0$、電磁場と振動子との間の結合強度f、およびその減衰$\delta$に依存する。半導体において、主な固有振動子または共鳴は、光学フォノン、それらのイオン化連続体およびより高いバンド間遷移またはプラズモンを含む励起子である。彼らはいくつかの詳細で説明されます。

7〜10章。これらの共鳴の光学的性質の多くの基本的特徴は類似していると我々は予想することができる。したがって、一般的な方法で、モデル振動子の集団の光学的性質を最初に議論することは合理的です。4〜6章の結果を使用することによって、11〜15章で、半導体の光学的性質への非常に単純で直接的なアクセスを得ることができます。

古典力学と電気力学の観点からモデル振動子の集合体の光学的性質を扱うことは、結果的に、非常に狭い等式性をもたらすことが判明した。これは複素誘電関数または屈折率のスペクトルに特に当てはまる。または反射と透過のスペクトルの。4つすべてが密接に関連しています(第6章)。したがって、しばらくの間この古典的なアプローチに従い、量子力学が適用された場合にどのような修正が行われるかを適切な場所で説明します。
これらのモデル発振器は、ローレンツ発振器として知られています。これらのローレンツ振動子の扱い、あるいは有限の導電率が含まれる場合(例えば、(2.24)を参照)、チャップスで述べられているような光学および固体物理学に関する多くの教科書にあります。[63H1,72W1]も参照。

第版の序文で述べたように、現在光学特性、特に半導体の電子システムの光学特性を光学式または半導体のブロッホ方程式で記述する傾向があります。
物理学がかなり複雑な数学的定式化の背後に隠されているか、それは数値解法で部分的に消えるという大きなリスクです。したがって、ローレンツ発振器の単純で直感的に明確な概念から始めますが、読者がそれに慣れることができるように第章で他の概念を提示します。
ここで、発振器のアンサンブルの光学特性について考えます。カップリングされていないオシレータの最も単純なケースから始めて、次の章のさまざまなステップで概念を洗練します。

\subsection{運動方程式と誘電関数}%サブセクション

我々は、同一の、結合されていない調和振動子のアンサンブルがあると仮定する。簡単にするために、図4.1aに示すように、格子定数aを有する光伝搬方向に周期的な一次元アレイを選択する。これらの高調波発振器はすべて同じ固有振動数ω0を有する。ダンピングを無視すると
そのとき、ω’は力学モデルで質量とバネの力定数βによって次のように表すことができます。

\subsubsection{name}%サブサブセクション

\if0
\begin{figure}[h]%画像はじめ[ページ上部]
 \centering%中心揃え
 \includegraphics[clip,width=9cm]{figure}%[切り取り、サイズ指定]ファイル名
 \caption{caption.}%グラフタイトル.
 \label{label}%グラフ識別子
\end{figure}%画像おわり
\fi

\if0
\begin{equation}%数式はじめ
 \Delta E=\frac{\mathrm{\hbar}^{2}\uppi^{2}}{2mL^{2}}%\frac{a}{b}->a/b
 \label{eq_qw}%式識別子
\end{equation}%数式おわり
\fi

\if0
\begin{table}[ht]%表はじめ
 \centering
 \caption{実験1の実験条件.}
 \begin{tabular}{lc}\hline
  \multicolumn{2}{c}{実験条件} \\ \hline
  中心波長 & 435 nm            \\
  露光時間 & 0.1 秒            \\
  積算回数 & 1 回              \\
  試料温度 & 19 ${}^\circ$C    \\ \hline
 \end{tabular}
 \label{col_1}%識別子
\end{table}%表おわり
\fi

\if0
\begin{figure}[ht]%画像はじめ
 \centering%中央揃え
 \begin{tabular}{c}%横ならべ
  %1
  \begin{minipage}{0.5\hsize}%占有割合指定
   \centering
   \includegraphics[clip, width=4.5cm,height=3cm]{figure}
   \hspace{1.6cm} [1]サブタイトル%サブタイトル1
  \end{minipage}

  %2
  \begin{minipage}{0.33\hsize}
   \centering
   \includegraphics[clip, width=4.5cm]{figure}
   \hspace{1.6cm} [2]サブタイトル%サブタイトル2
  \end{minipage}
 \end{tabular}%画像ならべおわり
 \caption{caption.}%グラフタイトル
 \label{label}%識別子

\end{figure}
\fi
\end{document}%本文おわり
